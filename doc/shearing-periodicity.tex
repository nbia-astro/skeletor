\documentclass[aps,pre,notitlepage,amsmath,amssymb,amsfonts,nobibnotes,nofootinbib,superscriptaddress,onecolumn,a4paper,10pt]{revtex4-1}
\setcitestyle{authoryear,round}
\bibliographystyle{apsrev4-1}
% Use the same section numbering as ApJ
% See http://tex.stackexchange.com/questions/87131/numbering-sections
\renewcommand{\thesection}{\arabic{section}}
\renewcommand{\thesubsection}{\thesection.\arabic{subsection}}
\renewcommand{\thesubsubsection}{\thesubsection.\arabic{subsubsection}}
% Fix references
\makeatletter
\def\p@subsection{}
\def\p@subsubsection{}
\makeatother

\usepackage{microtype}
\usepackage{lmodern}
\usepackage[T1]{fontenc}
\usepackage[utf8]{inputenc}
\usepackage[english]{babel}
% \usepackage{mathtools}
% \usepackage{graphicx}

% \usepackage[hidelinks]{hyperref}
\usepackage{cleveref}

\usepackage{bm}
\let\vec\bm{}

\usepackage{mleftright}
\let\left\mleft{}
\let\right\mright{}

% Derivatives
\newcommand{\pder}[2]{\frac{\partial#1}{\partial#2}}
\newcommand{\pdder}[2]{\frac{\partial^2#1}{\partial#2^2}}
\newcommand{\der}[2]{\frac{d#1}{d#2}}
\newcommand{\dder}[2]{\frac{d^2#1}{d#2^2}}
\newcommand{\pader}[2]{\partial{#1}/\partial{#2}}
\newcommand{\pderc}[3]{\left.\frac{\partial#1}{\partial#2}\right|_{#3}}

% Unit vectors
\newcommand{\ex}{\vec{e}_x}
\newcommand{\ey}{\vec{e}_y}
\newcommand{\ez}{\vec{e}_z}

\newcommand{\tvec}[1]{\tilde{\vec{#1}}}

\begin{document}

\title{Notes on shearing periodicity}
\author{Tobias Heinemann}
\author{Thomas Berlok}
\affiliation{Niels Bohr International Academy}
\maketitle

\section{Particle dynamics in the shearing sheet}

In a frame rotating with a constant angular velocity, Newton's second law may
be written as
\begin{equation}
  \label{eq:eom}
  \ddot{\vec{r}} + 2\vec{\Omega}\times\dot{\vec{r}} + \nabla\psi = \vec{a}.
\end{equation}
Here, $\vec{a}$ is the acceleration due to the Lorentz force, which we will
discuss in \cref{sec:electric,sec:magnetic}. For now let us ignore $\vec{a}$
and focus on the left hand side.

In standard shearing sheet coordinates, the angular velocity of the rotating
frame is $\vec{\Omega}=\Omega\ez$ and the effective potential is
$\psi=S\Omega x^2 + \nu^2 z^2/2$. For numerical reasons it is advantageous to
adopt a coordinate system that is rotated clock-wise around the $z$-axis by 90
degrees ($x\to y, y\to-x$). In the new coordinate system, the angular velocity
is the same but the effective potential is now
\begin{equation}
  \label{eq:effective-potential}
  \psi = S\Omega y^2 + \frac{1}{2}\nu^2 z^2.
\end{equation}

The horizontal components of \cref{eq:eom} upon substitution of
\cref{eq:effective-potential} are given by
\begin{align}
  \ddot{x} - 2\Omega\dot{y} &= a_x \\
  \label{eq:eom-y}
  \ddot{y} + 2\Omega(\dot{x} + Sy) &= a_y.
\end{align}
It follows immediately that the system is invariant under translations along
$x$:
\begin{equation}
  \label{eq:trans-x}
  \begin{aligned}
    x &\to x + \Delta x \\
    y &\to y
  \end{aligned}
\end{equation}
The system is also invariant under translations along $y$ if such translations
are followed by a boost along $x$:
\begin{equation}
  \label{eq:trans-y}
  \begin{aligned}
    y &\to y + \Delta y \\
    x &\to x - St\Delta y
  \end{aligned}
\end{equation}
The symmetries \labelcref{eq:trans-x,eq:trans-y} imply that in spite of the
glaring inhomogeneity in \cref{eq:effective-potential}, the system is in a
certain sense still homogeneous in the horizontal directions. In the words of
\citet{Wisdom1988}, ``the dynamics of the [linearized] model are independent
of the choice of origin, except for a uniform shear that is a consequence of
the differential rotation in a Keplerian disk.'' The uniform shear in question
is given by
\begin{equation}
  \label{eq:shear-flow}
  \vec{u} = -Sx\ey.
\end{equation}
This flow has the very special property that it appears the same to every
observer that is locally at rest with respect to it. In Galilean invariant
theories, only flows that depend linearly on the coordinates have this
property.

In terms of the one-particle distribution function $f(\vec{r},\vec{v},t)$ with
$\vec{v}=\dot{\vec{r}}$, the system of equations \labelcref{eq:eom} for many
particles may equivalently be written as
\begin{equation}
  \label{eq:vlasov}
  \pder{f}{t} + \vec{v}\cdot\nabla f
  + (\vec{a} - 2\vec{\Omega}\times\vec{v} - \nabla\psi)\cdot\pder{f}{\vec{v}}
  = 0.
\end{equation}
Depending on whether $f$ is the microscopic or macroscopic distribution
function, \cref{eq:vlasov} is either the Klimontovich or the Vlasov equation.
The former is exact while the latter only holds if the plasma is
collisionless. For our present discussion the distinction is not important and
we will refer to \cref{eq:vlasov} as the Vlasov equation.

Let us introduce the \emph{relative} velocity $\tvec{v}$ through
\begin{equation}
  \label{eq:relative-velocity}
  \vec{v} = -Sy\ex + \tvec{v}.
\end{equation}
Because
\begin{equation}
  \left.\pder{}{y}\right|_{\vec{v}} =
  \left.\pder{}{y}\right|_{\tvec{v}} + S\pder{}{\tilde{v}_x},
\end{equation}
the Vlasov equation \labelcref{eq:vlasov} expressed in terms of the phase
space coordinates $(\vec{r},\tvec{v})$ is
\begin{equation}
  \label{eq:vlasov-relative}
  \mathcal{D}f + \tilde{\vec{v}}\cdot\nabla f
  + (\vec{a} - 2\vec{\Omega}\times\tvec{v} + S\tilde{v}_y\ex - \nu^2 z\ez)
  \cdot\pder{f}{\tvec{v}} = 0,
\end{equation}
where we have introduced the differential operator
\begin{equation}
  \label{eq:shearing-ddt}
  \mathcal{D} = \pder{}{t} - Sy\pder{}{x}.
\end{equation}
While equation \cref{eq:shearing-ddt} appears more complicated than
\cref{eq:vlasov}, it has the distinct advantage that it depends explicitly on
$y$ only through the differential operator $\pader{}{t}-Sy\,\pader{}{x}$. This
explicit $y$-dependence can be eliminated (in favor of an explicit
$t$-dependence) through a transformation to shearing coordinates, see
\cref{sec:shearing-coordinates}.

So far we have only considered the motion of particles in a prescribed
effective potential. Let us now turn to the electromagnetic force and the
field equations governing the electromagnetic field (i.e.\ Maxwell's
equations). As pointed out by \citet{LeBellac1973}, it is essential for this
discussion to consider the two distinct non-relativistic limits of Maxwell's
equations.

\subsection{Electric limit}\label{sec:electric}

In the electric limit, the magnetic field plays an entirely passive role and
the Lorentz acceleration is simply
\begin{equation}
  \label{eq:electric-accel}
  \vec{a} = \frac{e}{m}\vec{E}.
\end{equation}
The system of equations is closed with Gauss' law
\begin{equation}
  \epsilon_0\nabla\cdot\vec{E} = \sum e\int\!d^3v\,f.
\end{equation}

In the electric limit, the electric field is invariant under Galilean
transformations:
\begin{equation}
  \vec{E}' = \vec{E}
\end{equation}
From this it follows that \cref{eq:vlasov} with $\vec{a}$ as given in
\cref{eq:electric-accel} is manifestly invariant under
\cref{eq:trans-x,eq:trans-y}.

\subsection{Magnetic limit}\label{sec:magnetic}

The magnetic field is not passive in this limit and the Lorentz acceleration
is given by
\begin{equation}
  \label{eq:magnetic-accel}
  \vec{a} = \frac{e}{m}(\vec{E} + \vec{v}\times\vec{B}).
\end{equation}
The system of equations is closed with Ampère's law
\begin{equation}
  \label{eq:ampere}
  \mu_0^{-1}\nabla\times\vec{B} = \sum e\int\!d^3v\,f\vec{v}
\end{equation}
and Faraday's law
\begin{equation}
  \label{eq:faraday}
  \pder{\vec{B}}{t} + \nabla\times\vec{E} = 0.
\end{equation}

The transformation law for the electromagnetic field in this limit is
\begin{equation}
  \label{eq:magnetic-transform}
  \begin{aligned}
    \vec{E}' &= \vec{E} + \Delta\vec{v}\times\vec{B} \\
    \vec{B}' &= \vec{B},
  \end{aligned}
\end{equation}
where $\Delta\vec{v}$ is the boost velocity. Since the transformation law for
the velocity itself is $\vec{v}'=\vec{v}-\Delta\vec{v}$, it is clear that the
Lorentz acceleration as given in \cref{eq:magnetic-accel} is Galilean
invariant. From \cref{eq:magnetic-transform} it also follows that the plasma
has to be strictly neutral in the magnetic limit. This is because if $\vec{B}$
is invariant (meaning $\vec{B}'=\vec{B}$), then so is its curl and hence the
left hand side of Ampère's law \labelcref{eq:ampere}. The right hand side is,
however, not invariant in general. This can only be reconciled by demanding
that
\begin{equation}
  \label{eq:strict-neutrality}
  \sum e\int\!d^3v\,f = 0,
\end{equation}
i.e.\ by demanding that the plasma is \emph{strictly} neutral. In this case
the right hand side of \cref{eq:ampere} is also invariant.

Let us now express \cref{eq:magnetic-accel,eq:ampere,eq:faraday} in terms of
the relative velocity $\tvec{v}$ defined in \cref{eq:relative-velocity}. This
will cast the equations in a form that is manifestly invariant under
\labelcref{eq:trans-x,eq:trans-y}. We start with Ampère's law
\labelcref{eq:ampere}. From \cref{eq:strict-neutrality} it follows that
\begin{equation}
  \mu_0^{-1}\nabla\times\vec{B} = \sum e\int\!d^3v\,f\tvec{v}.
\end{equation}
That is all. Next we consider the Lorentz acceleration defined in
\cref{eq:magnetic-accel}, which may be written as
\begin{equation}
  \vec{a} = \frac{e}{m}(\tvec{E} + \tvec{v}\times\vec{B}),
\end{equation}
where we have introduced the \emph{relative} electric field
\begin{equation}
  \tvec{E} = \vec{E} - Sy\ex\times\vec{B}.
\end{equation}
Expressing Faraday's law \labelcref{eq:faraday} in terms of $\tvec{E}$ yields
\begin{equation}
  \label{eq:faraday-relative}
  \mathcal{D}\vec{B} + SB_y\ex + \nabla\times\tvec{E} = 0,
\end{equation}
where the differential operator $\mathcal{D}$ is defined in
\cref{eq:shearing-ddt}. This form of Faraday's law makes explicit that the
magnetic field is advected (through $\mathcal{D}$) and stretched by the
uniform shear flow.

\section{Shearing coordinates}\label{sec:shearing-coordinates}

Both the Vlasov equation and Faraday's law in the form of
\cref{eq:vlasov-relative,eq:faraday-relative}, respectively, do not explicitly
depend on $x$. They do explicitly depend on $y$ in exactly the same way. This
$y$-dependence can, however, be eliminated through a transformation from
``laboratory coordinates'' ($x$, $y$, $t$) to shearing coordinates ($x'$,
$y'$, $t'$) defined by\footnote{We will often suppress the $z$-coordinate
  because it does not affect this discussion in any way.}
\begin{equation}
  \label{eq:shearing-coordinates}
  x' = x + Sty,\quad y' = y,\quad t' = t.
\end{equation}
From this it follows that
\begin{equation}
  \pder{}{x} = \pder{}{x'},\quad
  \pder{}{y} = \pder{}{y'} + St'\pder{}{x'},\quad
  \pder{}{t} - Sy\pder{}{x} = \pder{}{t'}.
\end{equation}
The second equality shows that an explicit time-dependence arises whenever a
partial derivative with respect to $y$ is taken. On the plus side, the third
equality shows that the explicit $y$-dependence is indeed eliminated.

In shearing coordinates, the system is thus homogeneous in the horizontal
directions. This means that we can impose periodic boundary conditions in
shearing coordinates. In turn this means that we can expand any field, say
$\tvec{E}$ in a double Fourier series:
\begin{equation}
  \tvec{E}(x',y',t') = \sum_{k_x',k_y'}\tvec{E}(k_x',k_y',t')
  e^{ik_x' x' + ik_y' y'}
\end{equation}
From \cref{eq:shearing-coordinates} it follows that
\begin{equation}
  k_x' x' + k_y' y' = k_x x + k_y y,
\end{equation}
where the laboratory frame wave numbers
\begin{align}
  k_x &= k_x' \\
  k_y &= k_y' + St' k_x'.
\end{align}
In the ``laboratory coordinates'' ($x$, $y$, $t$) we may thus express
$\tvec{E}$ as a sum of ``sheared disturbances'':
\begin{equation}
  \tvec{E}(x,y,t) = \sum_{k_x',k_y'}\tvec{E}(k_x',k_y',t)
  e^{ik_x x + ik_y(t) y},
\end{equation}
where we have indicated the explicit time-dependence of $k_y$. Because of this
time-dependence, the relative electric field is not periodic but
\emph{shearing-periodic} in laboratory coordinates.

\bibliography{shearing-periodicity}

\end{document}
