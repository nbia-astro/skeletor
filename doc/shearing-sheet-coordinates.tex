\documentclass[aps,pre,notitlepage,amsmath,amssymb,amsfonts,nobibnotes,nofootinbib,superscriptaddress,onecolumn,a4paper,10pt]{revtex4-1}
\setcitestyle{authoryear,round}
\bibliographystyle{apsrev4-1}
% Use the same section numbering as ApJ
% See http://tex.stackexchange.com/questions/87131/numbering-sections
\renewcommand{\thesection}{\arabic{section}}
\renewcommand{\thesubsection}{\thesection.\arabic{subsection}}
\renewcommand{\thesubsubsection}{\thesubsection.\arabic{subsubsection}}
% Fix references
\makeatletter
\def\p@subsection{}
\def\p@subsubsection{}
\makeatother

\usepackage{microtype}
\usepackage{lmodern}
\usepackage[T1]{fontenc}
\usepackage[utf8]{inputenc}
\usepackage[english]{babel}
% \usepackage{mathtools}
% \usepackage{graphicx}

% \usepackage[hidelinks]{hyperref}
\usepackage{cleveref}

\usepackage{bm}
\let\vec\bm{}

\usepackage{mleftright}
\let\left\mleft{}
\let\right\mright{}

% Derivatives
\newcommand{\pder}[2]{\frac{\partial#1}{\partial#2}}
\newcommand{\pdder}[2]{\frac{\partial^2#1}{\partial#2^2}}
\newcommand{\der}[2]{\frac{d#1}{d#2}}
\newcommand{\dder}[2]{\frac{d^2#1}{d#2^2}}
\newcommand{\pader}[2]{\partial{#1}/\partial{#2}}
\newcommand{\pderc}[3]{\left.\frac{\partial#1}{\partial#2}\right|_{#3}}

% Unit vectors
\newcommand{\ex}{\vec{e}_x}
\newcommand{\ey}{\vec{e}_y}
\newcommand{\ez}{\vec{e}_z}


\begin{document}

\section{Effective electromagnetic fields}
We start by a simple rederivation of the equation of motion.
The equation of motion is
\begin{equation}
  \label{eq:eom}
  \dot{\vec{v}} + 2\vec{\Omega}\times\vec{v} + \nabla\psi =
  \frac{e}{m}(\vec{E} + \vec{v}\times\vec{B}).
\end{equation}
Note that by introducing the ``effective'' electromagnetic fields
$\vec{E}_\ast$ and $\vec{B}_\ast$ via
\begin{equation}
  \frac{e}{m}\vec{E}_\ast = \frac{e}{m}\vec{E} - \nabla\psi
\end{equation}
and
\begin{equation}
  \frac{e}{m}\vec{B}_\ast = \frac{e}{m}\vec{B} + 2\vec{\Omega},
\end{equation}
we can rewrite \cref{eq:eom} as
\begin{equation}
  \dot{\vec{v}} =
  \frac{e}{m}(\vec{E}_\ast + \vec{v}\times\vec{B}_\ast).
\end{equation}

\section{Standard coordinates}
In standard coordinates the tidal potential is given by
\begin{equation}
  \label{eq:tidal-potential}
  \psi = -q\Omega^2 x^2 + \frac{1}{2}\nu^2 z^2.
\end{equation}
Assuming that in equilibrium, the \emph{horizontal} force balance is such that
the plasma is rotationally supported against gravity and the magnetic field is
frozen into the bulk flow $\vec{u}=\int\!f\vec{v}\,d^3v$ (and hence
$\vec{E}+\vec{u}\times\vec{B}=0$), we must have
\begin{equation}
  \label{eq:gravito-centrifugal-balance}
  2\vec{\Omega}\times\vec{u} + \nabla_\perp\psi = 0.
\end{equation}
From \cref{eq:tidal-potential} it follows that the horizontal gradient of
$\psi$ is
\begin{equation}
   \nabla_\perp\psi = -2q\Omega^2 x\ex.
\end{equation}
Given this, solving \cref{eq:gravito-centrifugal-balance} for $\vec{u}$ yields
\begin{equation}
  \label{eq:common-shear-flow}
  \vec{u} = Sx\ey,
\end{equation}
where
\begin{equation}
  S = -q\Omega
\end{equation}
is the rate of shear (note also that $\nabla\times\vec{u}=S\ez$).

\section{Shear along $y$}
We use the same equation of motion but the tidal potential is taken to be
given by (replacing $x$ by $y$)
\begin{equation}
  \psi = -q\Omega^2 y^2 + \frac{1}{2}\nu^2 z^2  \ .
\end{equation}
In this case,
\begin{equation}
  2 \vec{\Omega} \times \vec{u}_s = -\nabla \psi
  \ .
\end{equation}
\begin{equation}
  2 \Omega (\ey u_x - \ex u_y) = 2q\Omega^2 y \ey -\nu^2 z \ez
  \ .
\end{equation}
so that
\begin{equation}
\vec{u}_s = q \Omega y \ex = -Sy\ex
\end{equation}
Note that our $x$-coordinate is oriented in the opposite direction to the
standard $x$-coordinate. So the physical background motion is in the same
direction regardless of the coordinate system. We now plug
\begin{equation}
\nabla \psi = - 2 \vec{\Omega} \times \vec{u}_s
= 2Sy\vec{\Omega} \times \ex
\end{equation}
into the equation of motion
\begin{equation}
  \dot{\vec{v}} + 2\vec{\Omega}\times\vec{v} + 2Sy\vec{\Omega} \times \ex =
  \frac{e}{m}\vec{E} + \frac{e}{m}\vec{v}\times\vec{B},
\end{equation}
and rewrite
\begin{equation}
  \dot{\vec{v}} =
  \frac{e}{m}\left(\vec{E} + Sy \ex \times 2m\vec{\Omega}/e\right) +
  \frac{e}{m}\vec{v}\times \left( \vec{B} + 2m\vec{\Omega}/e \right)
\end{equation}
We conclude that $\vec{B}_\ast$ is the same as in the standard case and that
$\vec{E}_\ast$ is given by
\begin{equation}
  \vec{E}_\ast = \vec{E} + Sy \ex \times 2m\vec{\Omega}/e
\end{equation}

\section{Implementation notes}
In the code we have (so far) $\vec{\Omega} = \Omega \ez$. In this case
\begin{equation}
  \vec{E}_\ast = \vec{E} - Sx \frac{2m\Omega}{e} \ex
\end{equation}
for the standard case and
\begin{equation}
  \vec{E}_\ast = \vec{E} - Sy \frac{2m\Omega}{e} \ey
\end{equation}
for the shear along $y$ case.

\section{Epicycles}
\begin{equation}
  \ddot{\vec{r}} + 2\vec{\Omega}\times\dot{\vec{r}} + \nabla\psi = 0.
\end{equation}
Assuming $\vec{\Omega} = \Omega \ez$
For the standard case
\begin{equation}
\nabla \psi = -2Sx\vec{\Omega}\times \ey = 2Sx\Omega \ex
\end{equation}

\begin{align}
  \ddot{x} - 2 \Omega \dot{y} + 2 S \Omega x = 0 \\
  \ddot{y} + 2 \Omega \dot{x} = 0 \ ,
\end{align}
The analytical solution in our other set of notes is reproduced here for
convenience
\begin{align}
  x(t) &= A\cos(\kappa t + \phi) + x_0, \\
  y(t) &= -\frac{2\Omega}{\kappa} A\sin(\kappa t + \phi) + y_0 + S t x_0,
\end{align}
where
\begin{equation}
  \kappa = \sqrt{2\Omega(2\Omega + S)} \ .
\end{equation}
For the new coordinate system the equations are instead
\begin{equation}
\nabla \psi = 2Sy\vec{\Omega} \times \ex = 2Sy \Omega\ey
\end{equation}

\begin{align}
  \ddot{x} - 2 \Omega \dot{y} = 0 \\
  \ddot{y} + 2 \Omega \dot{x} + 2 S \Omega y = 0 \ ,
\end{align}
We find that the analytical solution in this case is
\begin{align}
  x(t) &= +\frac{2\Omega}{\kappa} A\sin(\kappa t + \phi) + x_0 - S t y_0 \\
  y(t) &= A\cos(\kappa t + \phi) + y_0.
\end{align}
\end{document}
